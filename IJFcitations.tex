\documentclass[11pt,a4paper]{elsarticle} %review=doublespace preprint=single 5p=2 column
%%% Begin My package additions %%%%%%%%%%%%%%%%%%%
\usepackage[hyphens]{url}

  \journal{International Journal of Forecasting} % Sets Journal name


\usepackage{lineno} % add
\providecommand{\tightlist}{%
  \setlength{\itemsep}{0pt}\setlength{\parskip}{0pt}}

\bibliographystyle{elsarticle-harv}
\biboptions{sort&compress} % For natbib
\usepackage{graphicx}
\usepackage{booktabs} % book-quality tables
%%%%%%%%%%%%%%%% end my additions to header

\usepackage[T1]{fontenc}
\usepackage{lmodern}
\usepackage{amssymb,amsmath}
\usepackage{ifxetex,ifluatex}
\usepackage{fixltx2e} % provides \textsubscript
% use upquote if available, for straight quotes in verbatim environments
\IfFileExists{upquote.sty}{\usepackage{upquote}}{}
\ifnum 0\ifxetex 1\fi\ifluatex 1\fi=0 % if pdftex
  \usepackage[utf8]{inputenc}
\else % if luatex or xelatex
  \usepackage{fontspec}
  \ifxetex
    \usepackage{xltxtra,xunicode}
  \fi
  \defaultfontfeatures{Mapping=tex-text,Scale=MatchLowercase}
  \newcommand{\euro}{€}
\fi
% use microtype if available
\IfFileExists{microtype.sty}{\usepackage{microtype}}{}
\usepackage[margin=1.5cm]{geometry}
\ifxetex
  \usepackage[setpagesize=false, % page size defined by xetex
              unicode=false, % unicode breaks when used with xetex
              xetex]{hyperref}
\else
  \usepackage[unicode=true]{hyperref}
\fi
\hypersetup{breaklinks=true,
            bookmarks=true,
            pdfauthor={},
            pdftitle={A retrospective analysis of International Journal of Forecasting based on bibliometric analysis and knowledge flow analysis},
            colorlinks=false,
            urlcolor=blue,
            linkcolor=magenta,
            pdfborder={0 0 0}}
\urlstyle{same}  % don't use monospace font for urls

\setcounter{secnumdepth}{5}
% Pandoc toggle for numbering sections (defaults to be off)
% Pandoc header
%% FONT and LAYOUT
\usepackage{mathpazo}
\usepackage{parskip}

%% CAPTIONS
\usepackage{caption}
\DeclareCaptionStyle{italic}[justification=centering]
 {labelfont={bf},textfont={it},labelsep=colon}
\captionsetup[figure]{style=italic,format=hang,singlelinecheck=true}
\captionsetup[table]{style=italic,format=hang,singlelinecheck=true}

%% MATHS
\usepackage{bm}
\allowdisplaybreaks

%% GRAPHICS
\usepackage[section]{placeins}
\setcounter{topnumber}{2}
\setcounter{bottomnumber}{2}
\setcounter{totalnumber}{4}
\renewcommand{\topfraction}{0.85}
\renewcommand{\bottomfraction}{0.85}
\renewcommand{\textfraction}{0.15}
\renewcommand{\floatpagefraction}{0.7}



\begin{document}
\begin{frontmatter}

  \title{A retrospective analysis of International Journal of Forecasting based
on bibliometric analysis and knowledge flow analysis}
    \author[nau]{Dejian Yu}
   \ead{yudejian62@126.com} 
  
    \author[monash]{Rob J Hyndman}
   \ead{Rob.Hyndman@monash.edu} 
  
    \author[zufe]{Shunshun Shi\corref{c1}}
   \ead{shishunshun1993@126.com} 
   \cortext[c1]{Corresponding Author}
      \address[nau]{Business School, Nanjing Audit University, Nanjing 211815, China}
    \address[monash]{Department of Econometrics \& Business Statistics, Clayton, VIC 3800,
Australia}
    \address[zufe]{Information School, Zhejiang University of Finance and Economics,
Hangzhou 310016, China}
  
  \begin{abstract}
  As one of the leading journals in the forecasting field, International
  Journal of Forecasting (IJF) has continually contributed to the field
  over 30 years. In this study, a retrospective analysis is conducted to
  evaluate the merits of IJF from 1985 to 2018 based on the raw data
  harvested from Scopus. The retrospective analysis consists of two parts.
  One is a bibliometric analysis of IJF based on a science mapping
  technique which focuses on the level of paper, author and country. A
  citation network, a co-citation network, a co-author network and a
  country co-occurrence network are mapped based on IJF's publications.
  Multiple relationships between different scientific entities can be
  explored from these networks. The other is a knowledge flow analysis
  with measures of Interdisciplinary Research (IDR) based on the citing
  papers (i.e., the papers cite IJF's publications). The discipline
  distribution of citing papers can be obtained with the citation
  relationship between IJF's publications and their citing papers.
  Besides, a dynamic analysis of citing disciplines over time uncovers the
  changes of knowledge diffusion starting from IJF's publications. A
  deeper investigation about the knowledge flow outside of the forecasting
  journals is conducted for better elaborating the performance of IJF in
  knowledge diffusion outside the forecasting field. These two kinds of
  analyses directly depict a landscape about the development track of IJF
  based on the endorsements it obtained and the knowledge it spread
  abroad.
  \end{abstract}
   \begin{keyword} citations, forecasting, bibliometric, knowledge flow\end{keyword}
 \end{frontmatter}

\newpage

\section{Introduction}\label{introduction}

(Content to be added)

\section{Bibliometric analysis of
IJF}\label{bibliometric-analysis-of-ijf}

(Content to be added)

\subsection{Basic statistics of IJF}\label{basic-statistics-of-ijf}

In this subsection, a basic statistics of IJF about the publications and
citations is provided. The analysis about IJF publications focuses on
describing how the number of publications changed annually from 1985 to
2018. The analysis about IJF citations focuses on revealing the dynamics
of citations that the IJF publications received annually from 1986 to
2019. Note that the number of citations in 2019 is for reference only
because the citation window remains nine months left and the citation
data will increase over time. Besides, a citation distribution analysis
about the IJF publications is conducted. In this citation distribution
analysis, IJF publications are partitioned based on the number of
citations they received. For example, the publications received
citations more 20 but less than 30 are partitioned as the same group.
Different groups contain different publications based on the number of
citations, and the groups are ranked with an increasing citation order.

\subsection{Citation network and co-citation
network}\label{citation-network-and-co-citation-network}

In this subsection, we first extract some most highly cited IJF
publications with the information about the citation number, author
names, publication year, etc. Then a citation network is mapped based on
the citation relationship between IJF publications and their citations.
A citation network consists of some nodes and links, where some nodes
are connected with links and some are isolated. A node represents a
publication and a link represents the citation relationship between any
two connected publications. The number of citations a publication
received is denoted as the size of node. Through this citation network,
the citation relationships between any connected publications can be
visually obtained and deeper investigations like highly cited
publications are cited by what papers can be conducted easily. The
co-citation network focuses on the co-citation relationships between IJF
publications and their citations. The performance of co-citation network
is similar to that of citation network. Based on the co-citation
network, the co-citation relationships between any co-cited publications
can be identified.

\subsection{Authors analysis in IJF and co-author
network}\label{authors-analysis-in-ijf-and-co-author-network}

In this subsection, we first extract some most prolific authors in IJF.
Note that only the first author is considered. Then we set two
indicators. One is local citation which means the number of citations
received from the IJF publications. The other is global citation which
indicates the number of citations received outside the IJF publications.
Through the analysis of local citation, the popular authors who are
welcomed by IJF authors can be identified, while in the analysis of
global citation, the popular authors who are welcomed by authors from
various journals can be obtained. Besides, a co-author network is mapped
based on the co-author relationship between any connected authors.

\subsection{Country analysis and country co-occurrence
network}\label{country-analysis-and-country-co-occurrence-network}

In this subsection, we extract some prolific countries and most
attractive countries according to their number of publications and
citations. Based on this analysis, we can identify what countries are
the biggest providers of IJF and what countries possess the highest
authority in IJF. Moreover, a dynamic analysis about the most prolific
countries over time can be conducted and the result will be visualized
in a world map. Besides, a country co-occurrence network is provided to
express the collaboration relationships among different countries.

\section{Knowledge flow analysis of
IJF}\label{knowledge-flow-analysis-of-ijf}

(Content to be added)

\subsection{Discipline distribution of citing
papers}\label{discipline-distribution-of-citing-papers}

In this subsection, citing papers (i.e., papers cite IJF publications)
are selected as the raw data. Statistics about the discipline
distribution of citing papers is provided to figure out that what
disciplines pay attention to the IJF publications. Also we provide the
ranking of highly citing disciplines.

\subsection{Dynamic of citing disciplines over
time}\label{dynamic-of-citing-disciplines-over-time}

IJF was set in 1985 and its initial aim was to take a
??multi-disciplinary perspective?? to all types of forecasting methods
(Robert, 2006). Under the investigation on hot topics in IJF, Robert
argued that the themes had developed over time with the change of
dominating application areas, and he agree with Makridakis??s viewpoint
(2006) that ??there is a need for the forecasting community to establish
a new research agenda based on new approaches to important problems??.
Therefore, in this subsection, we focus on the dynamic analysis of
citing disciplines over time to delineate the development track of IJF
from the angle of what disciplines it has attracted. The publication
years of citing papers will be sliced into several pieces to construct a
dynamic framework. The citation relationships between IJF publications
and their citing papers will be investigated in deep.

\subsubsection{Knowledge diffusion of IJF covering all citing
fields}\label{knowledge-diffusion-of-ijf-covering-all-citing-fields}

The data contains all citing papers, and the citing papers are grouped
based on several time slices. According to the categories predefined by
Scopus, we list the all categories where the citing papers belong, and
then analyze the results.

\subsubsection{Contrastive analysis of knowledge diffusion between IJF
and
JF}\label{contrastive-analysis-of-knowledge-diffusion-between-ijf-and-jf}

Journal of Forecasting (JF) is another reputable journal in the field of
forecasting, so we conduct a contrastive analysis of knowledge diffusion
between IJF and JF to further distinguish their differences in citation
relationships based on measures of Interdisciplinary Research. Robert
(2006) argued that IJF and JF paid more attention on papers published on
these two journals, and after a cross-citation examination, he concluded
there was little cross-fertilisation between these two journals and
other journals in the fields of business, economics and management
(BEM). In this study, we are more inclined to analyze the
cross-fertilisation from the angle of category. We explore what
categories have been attracted by JIF and JF and what is the differences
between the categories attracted by JIF and the categories attracted by
JF. In addition, the strength of the citation connection belonging to
each category is investigated based on citation relationships.

\subsubsection{Knowledge diffusion of IJF outside the forecasting
field}\label{knowledge-diffusion-of-ijf-outside-the-forecasting-field}

In this subsection, the raw data is limited to the citing papers
belonging to the fields which are outside the forecasting field.
According to the aim of the International Institute of Forecasters
(IIF), IJF and JF are two primary contributing journals in the field of
forecasting, so the citing papers published in IJF and JF are excluded
in this part of work. Knowledge diffusion starting from the publications
of IJF and JF to the publications in the fields which are outside the
forecasting field is explored. Journals in the fields of BEM are
important venues to publish forecasting application papers, so these
journals are the key objectives. Moreover, citing papers from other
journals are also under investigation, and we believe valuable
information can be found in the non-BEM journals.

\section{Conclusion and discussion}\label{conclusion-and-discussion}

(Content to be added)

\section*{References}\label{references}
\addcontentsline{toc}{section}{References}

\end{document}


